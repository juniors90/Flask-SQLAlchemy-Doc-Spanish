%%
%% This is file `Thesis.cls', based on 'ECSthesis.cls', by Steve R. Gunn
%% generated with the docstrip utility.
%%
%% Created by Steve R. Gunn, modified by Sunil Patel: www.sunilpatel.co.uk
\documentclass[10pt,letterpaper,notumble]{leaflet}

\usepackage[T1]{fontenc}
\usepackage[utf8]{inputenc}
\usepackage{listingsutf8}
\usepackage[spanish]{babel}
\usepackage{microtype}
\usepackage{blindtext} 
\usepackage{biolinum} 
\renewcommand\rmdefault{\sfdefault}% Verwende serifenlose Schrift 
\usepackage{mwe}% Dummy Bilder 
\usepackage{graphicx}
\usepackage{verbatim}
\usepackage{xcolor}
	\definecolor{WildStrawberry}{rgb}{1.0, 0.26, 0.64}
	\definecolor{wildstrawberry}{rgb}{1.0, 0.26, 0.64}
	\definecolor{Mulberry}{rgb}{0.77, 0.29, 0.55}
	\definecolor{LimeGreen}{rgb}{0.2, 0.8, 0.2}
	\definecolor{LincolnGreen}{rgb}{0.11, 0.35, 0.02}
	\definecolor{blue}{rgb}{0.0, 0.0, 1.0}
	\definecolor{forestgreen(traditional)}{rgb}{0.0, 0.27, 0.13}
\usepackage[colorlinks,citecolor=blue,urlcolor=blue]{hyperref}


\AddToBackground{5}{\put(0,0){\textcolor{blue!5}{\rule{\paperwidth}{\paperheight}}}}%

\AddToBackground{5}{% Fondo de la página pequeña 1
	\put(\LenToUnit{0.05\paperwidth},\LenToUnit{0.875\paperheight}){\includegraphics[width=\LenToUnit{0.90\paperwidth},height=\LenToUnit{0.1\paperheight}]{../img/flask-sqlalchemy-title.png}}
}

\title{Quik Reference} 
\author{Ferreira Juan David} 
\date{\today} % Experiment 1

\begin{document}
	\mbox{}
	\thispagestyle{empty}% Keine Seitenzahlen 
	
	
	\clearpage % end the column
	
	\makebox[\linewidth][l]{
		\begin{minipage}{1.65\linewidth}
			
			\section{Clase Query}
			
			También es posible personalizar lo que está disponible para su uso en la propiedad especial \texttt{query} de los modelos. Por ejemplo, proporcionando un método \texttt{get\_or}:
			
			\lstset{inputencoding=utf8/latin1,
				frame=lines,
				label={lst:code_direct},
				basicstyle=\footnotesize,
				showstringspaces=false  
			}
			\lstinputlisting[language=Python, firstline=60, lastline=69]{../python/05-Multiple-Databases-with-Binds.py}
			
			Y ahora todas las consultas ejecutadas desde la querypropiedad especial en los modelos \texttt{Flask-SQLAlchemy} pueden usar el \texttt{get\_ormétodo} como parte de sus consultas. Todas las relaciones definidas con \texttt{db.relationship} (pero no \texttt{sqlalchemy.orm.relationship()}) también se proporcionarán con esta funcionalidad.
			
			También es posible definir una clase de consulta personalizada para relaciones individuales, proporcionando la palabra clave \texttt{query\_class} en la definición. Esto funciona con ambos \texttt{db.relationship} y \texttt{sqlalchemy.relationship}:
			
			
			\lstinputlisting[language=Python, firstline=71, lastline=72]{../python/05-Multiple-Databases-with-Binds.py}
			
			\textbf{Nota}: Si una clase de consulta se define en una relación, tendrá prioridad sobre la clase de consulta adjunta a su modelo correspondiente.
			
			También es posible definir una clase de consulta específica para modelos individuales anulando el atributo \texttt{query\_class} de clase en el modelo:
			
			
			\lstinputlisting[language=Python, firstline=74, lastline=75]{../python/05-Multiple-Databases-with-Binds.py}
			
			En este caso, el método \texttt{get\_or} solo estará disponible en consultas que se originen en \texttt{MyModel.query}.
			
			\vspace*{-0.5cm}
			
			\section{Metaclase del modelo} 
			
			Advertencia: Las metaclases son un tema avanzado y probablemente no necesites personalizarlas para lograr lo que deseas. Se documenta principalmente aquí para mostrar cómo deshabilitar la generación de nombres de tablas.
			
			La metaclase del modelo es responsable de configurar los componentes internos de \texttt{SQLAlchemy} al definir las subclases del modelo. \texttt{Flask-SQLAlchemy} agrega algunos comportamientos adicionales a través de mixins; su metaclase predeterminada \texttt{DefaultMeta}, los hereda todos.
			
			BindMetaMixin: \texttt{\_\_bind\_key\_\_} se extrae de la clase y se aplica a la tabla. Consulte Varias bases de datos con enlaces.
			
		\end{minipage}
	}
	\clearpage
	{\hspace{4.5cm}
	\makebox[\linewidth][l]{
		\begin{minipage}{1.65\linewidth}
			
			\texttt{NameMetaMixin}: Si el modelo no especifica una \texttt{\_\_tablename\_\_} clave principal pero sí especifica, se genera un nombre automáticamente.
			
			Puede agregar sus propios comportamientos definiendo su propia metaclase y creando la base declarativa usted mismo. Asegúrese de seguir heredando de los mixins que desee (o simplemente herede de la metaclase predeterminada).
			
			Pasar una clase base declarativa en lugar de una clase base de modelo simple, como se muestra arriba, \texttt{base\_class} hará que \texttt{Flask-SQLAlchemy} use esta base en lugar de construir una con la metaclase predeterminada
			
			\lstset{inputencoding=utf8/latin1,
				frame=lines,
				label={lst:code_direct},
				basicstyle=\footnotesize,
				showstringspaces=false  
			}
			\lstinputlisting[language=Python, firstline=77, lastline=90]{../python/05-Multiple-Databases-with-Binds.py}
			
			También puede pasar cualquier otro argumento que desee \texttt{declarative\_base()} para personalizar la clase base según sea necesario.
			
			\textbf{Deshabilitar la generación de nombres de tablas}
			 
			Algunos proyectos prefieren configurar cada modelo \texttt{\_\_tablename\_\_} manualmente en lugar de depender de la detección y generación de \texttt{Flask-SQLAlchemy}. La generación de nombres de tablas se puede desactivar definiendo una \texttt{metaclase} personalizada.
			
			\lstinputlisting[language=Python, firstline=92, lastline=99]{../python/05-Multiple-Databases-with-Binds.py}
			
			Esto crea una base que aún admite la función \texttt{\_\_bind\_key\_\_} pero no genera nombres de tablas.
			
		\end{minipage}
		
	}}
	\clearpage
	
	\mbox{}
	
	\clearpage % end the spanned column
	
	\maketitle
	
	\begin{abstract}
		A partir de \texttt{Flask-SQLAlchemy 0.12} puede conectarse fácilmente a múltiples bases de datos. Para lograrlo, preconfigura \texttt{SQLAlchemy} para que admita múltiples ``enlaces''.
		
		¿Qué son las ataduras?. En \texttt{SQLAlchemy}, un enlace es algo que puede ejecutar sentencias \texttt{SQL} y generalmente es una conexión o un motor. En \texttt{Flask-SQLAlchemy}, los enlaces son siempre motores que se crean automáticamente entre bastidores. Luego, cada uno de estos motores se asocia con una clave corta (la clave de enlace). Esta clave se usa luego en el momento de la declaración del modelo para asociar un modelo con un motor específico.
		
		Si no se especifica ninguna clave de enlace para un modelo, se utiliza la conexión predeterminada en su lugar (según la configuración de \texttt{SQLALCHEMY\_DATABASE\_URI}).
	\end{abstract}
	\begin{center}
		\includegraphics[width=\textwidth]{../img/flask-sqlalchemy-logo.png}
	\end{center}
	
	\thispagestyle{empty}
	
	\clearpage
	\makebox[\linewidth][l]{
		\begin{minipage}{2.2\linewidth} 
				
			\vspace*{-0.4cm}
			
			\section{Customizing}
			
			\texttt{Flask-SQLAlchemy} define valores predeterminados sensibles. Sin embargo, a veces se necesita personalización. Hay varias formas de personalizar cómo se definen e interactúan los modelos.
			
			Estas personalizaciones se aplican en la creación del \texttt{SQLAlchemy} objeto y se extienden a todos los modelos derivados de su clase \texttt{Model}.
			
			\textbf{Clase Model}
			
			Todos los modelos de \texttt{SQLAlchemy} heredan de una clase base declarativa. Esto se expone como \texttt{db.Model} en \texttt{Flask-SQLAlchemy}, que amplían todos los modelos. Esto se puede personalizar subclasificando el predeterminado y pasando la clase personalizada a \texttt{model\_class}.
			
			El siguiente ejemplo le da a cada modelo una clave primaria entera o una clave externa para la herencia de tablas unidas.
			
			\textbf{Nota}: Las claves primarias enteras para todo no son necesariamente el mejor diseño de base de datos (eso depende de los requisitos de su proyecto), esto es solo un ejemplo.
			
			\lstset{inputencoding=utf8/latin1,
				frame=lines,
				label={lst:code_direct},
				basicstyle=\footnotesize,
				showstringspaces=false  
			}
			
			\lstinputlisting[language=Python, firstline=23, lastline=45]{../python/05-Multiple-Databases-with-Binds.py}
			
			\textbf{Model Mixins}
			
			Si el comportamiento solo es necesario en algunos modelos en lugar de en todos los modelos, use las clases \texttt{mixin} para personalizar solo esos modelos. Por ejemplo, si algunos modelos deben realizar un seguimiento de cuándo se crean o actualizan:
			
			\lstinputlisting[language=Python, firstline=47, lastline=58]{../python/05-Multiple-Databases-with-Binds.py}
			
			\vspace*{-0.5cm}
			 
		\end{minipage}
		
	}
	
\end{document}